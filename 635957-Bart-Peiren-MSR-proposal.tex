\documentclass{article}
\usepackage[british]{babel}
\usepackage{appendix}
\usepackage{hyperref}
\usepackage{csquotes}
% Use numeric citations and keep bibliography in citation order (appearance)
\usepackage[backend=biber,style=numeric,sorting=none]{biblatex}
\addbibresource{msr-proposal.bib}

\author{Bart Peiren}
\title{Research proposal:\\Explore and develop a route planner for people with limited mobility with OpenStreetMap}

\begin{document}
    \maketitle

\begin{abstract}
    People with limited mobility encounter obstacles which are not taken into account by common route planners.

    In 2023-2024 Amsterdam Intelligence at the Municipality of Amsterdam Intelligence created a route planner for
    people with reduced mobility. Their source code is still available 
    \cite{AmsterdamAITeamAccessible_Route_PlanningRepository}, but their blog post no longer is. 

    The data source in their implementation are 3D point clouds, obtained via LiDAR scans
    \cite{fiawoPuntenwolkenVanGedetailleerde2022}. While this is an ideal source, it might not be be available
    for every location nor be kept up-to-date.
    
    OpenStreetMap \cite{OpenStreetMap} offers a more limited, but freely and widely available map
    which might be sufficient for this goal. The goal is to verify whether its data suffices to build a 
    limited mobility route planner and if so, build such a route planner.
\end{abstract}


\section*{Introduction}

Common route planners create pedestrian routes for able bodied people. They do not take into account
obstacles which less mobile people might experience. This makes it harder for them to leave their home,
especially when travelling to unknown places.

While the data obtained via a LiDAR scan offers many possibilities, it also comes with two downsides: the whole 
environment where we want to plan a route needs to be mapped with special devices, and afterwards this data 
needs to be analysed so the useful data is extracted from it.

OpenStreetMap already provides a map for most of the world. It might not be detailed enough for all use, but 
most likely those areas will not be LiDAR scanned either. 

If we build an accessible route planner based on OpenStreetMap, in a later phase the LiDAR data can be processed
and based on it, OpenStreetMap could be updated automatically. In this way the LiDAR data serves a broader goal
rather than the limited few applications.


\section*{Literature review}

The website on Amsterdam Intelligence's work \cite{CreatingPersonalizedRoute} unfortunately is no longer 
available. Fortunately their source code remains available on GitHub
\cite{AmsterdamAITeamAccessible_Route_PlanningRepository} for further study. Their work was aimed at 
The Netherlands, so can not be widely reused abroad.

\citetitle{cohenRoutePlanningBlind2021} describes an implementation similiar what I want to achieve, but
for blind people. Planning a route for less mobile people could be similar in idea, but with different weights.

In \citeyear{biagiMappingAccessabilityOpenStreetMap2020} a study was conducted on different techniques of 
adding data in OpenStreetMap with respect to accessibility \cite{biagiMappingAccessabilityOpenStreetMap2020}.
This is important since OpenStreetMap depends on volunteers to gather and insert all the data. The easier
it is to add all the data required for accessibility, the more data will be made available, and the 
better an accessible route planner will be able to work.


\section*{Research methodology}

As we intend to to build a route planner for less mobile people, we should first define our precise target
audience. There are various forms of reduced mobility: wheelchair bound people, people using crutches temporarily 
or permanently, being less mobile due to age, blindness, \dots. Do we take all forms into account, or do we focus on a certain subgroup?

Once we know our target audience, we need to define their needs for an \textit{accessible route}. Several 
aspects impact a route:

\begin{itemize}
    \item What kind of slope is acceptable?
    \item What is the impact of different kinds of surface: concrete, paving stones, grass, mud, cobblestones, etc.
    \item Is the walkway separated from the road?
    \item How wide is the walkway?
    \item Is there a curb, and how high is it?
    \item Are there stairs? Do they include a ramp?
\end{itemize}

Using these requirements, we can assign different weights to the different aspects. Given that different aspects
impact different ableness differently, weights could differ for different populations.

When building the route planner, it will need to take these weights into account based on a chosen profile.

\section*{Expected results}

Planning routes based on commonly and freely available mapping data and tailored to a user's specific needs
makes it easier for them to travel around independently.

Combine this with a presentation tailored to their needs (e.g. auditory guidance for blind people) and their
dependence on other people or guidance dogs should be reduced and make them more confident to travel alone.

% \section*{Timeline}

\printbibliography

% \appendix
% \section{Appendix A}

    
\end{document}