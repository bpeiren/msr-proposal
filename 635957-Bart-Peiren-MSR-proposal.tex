\documentclass{article}
\usepackage[british]{babel}
\usepackage{appendix}
\usepackage{hyperref}
\usepackage{csquotes}
% Use numeric citations and keep bibliography in citation order (appearance)
\usepackage[backend=biber,style=numeric,sorting=none]{biblatex}
\addbibresource{msr-proposal.bib}

\author{Bart Peiren}
\title{Research proposal:\\Explore and develop a route planner for people with limited mobility with OpenStreetMap}

\begin{document}

\maketitle


\section*{Introduction}

People with limited mobility encounter obstacles which are not taken into account by common route planners.
Common route planners create pedestrian routes for able bodied people. They do not take into account
obstacles which less mobile people might experience. This makes it harder for them to leave their home,
especially when travelling to unknown places.

While the data obtained via a LiDAR scan offers many possibilities, it also comes with two downsides: the whole 
environment where we want to plan a route needs to be mapped with special devices, and afterwards this data 
needs to be analysed so the useful data is extracted from it.

OpenStreetMap \cite{OpenStreetMap} already provides a map for most of the world. It might not be detailed 
enough for all use, but most likely those areas will not be LiDAR scanned either. 

If we build an accessible route planner based on OpenStreetMap, in a later phase the LiDAR data can be processed
and based on it, OpenStreetMap could be updated automatically. In this way the LiDAR data serves a broader goal
rather than the limited few applications.


\section*{Literature review}

\begin{displayquote}[\textcite{kaselourisAccessibilityRoutingModel2025}]
  Accessibility in urban environments is a fundamental right of people with disabilities and should be a 
  key component of an inclusive society.
\end{displayquote}

\textcite{kaselourisAccessibilityRoutingModel2025} describe the mapping of an Athenian municipality and 
the building of a mobile application for accessible route planning. The mapping was done by performing
street audits in a limited area and further processed to arrive at data ready to use in their algoritm.

\textcite{gharebaghiUserSpecificRoutePlanning2021} researched using a fuzzy approach to determine 
accessible routes for \textit{people with motor disabilities} in Quebec. This research focuses more on
the algorithm to obtain a route than the data the algorithm is based on.

In 2023-2024 Amsterdam Intelligence at the Municipality of Amsterdam Intelligence created a route planner for
people with reduced mobility. Their website \cite{CreatingPersonalizedRoute} is no longer available, but 
fortunately their source code remains available on GitHub
\cite{AmsterdamAITeamAccessible_Route_PlanningRepository} for further study. Their work was aimed at 
The Netherlands, so can not be widely reused abroad. The data source in their implementation are 3D point clouds, obtained via LiDAR scans
\cite{fiawoPuntenwolkenVanGedetailleerde2022}. This method offers very detailed data, and makes precise
mapping of the environment possible, including walkway width, curb heights, as well as fixed obstacles 
such as bins and benches. While this is an ideal source, it is very resource intensive and might not be 
available for every location nor be kept up-to-date.

\textcite{cohenRoutePlanningBlind2021} describe an implementation of a route planner for blind people 
based on a weighted graph. Planning a route for less mobile people could be similar in idea, but with 
different weights.

\textcite{kocaaslanFindingOptimalRoute2024} performed a study in Turkey using the analytical hierarchy 
process to find accessible routes. They used existing maps where they assigned various criteria
a certain importance.

The data made available in OpenStreetMap is more limited than data available from LiDAR scans. It might also
be less detailed than other existing maps. However, it is freely and widely available, and as such more 
accessible. 
Though limited, it might be sufficient for our purpose. It offers a wide variety of functionality to map roads with 
their sidewalks. The degree of available detail in a given area decides how useful OSM could be for
accessible route planning in this area.

In \citeyear{biagiMappingAccessabilityOpenStreetMap2020} a study was conducted on different techniques of 
adding data in OpenStreetMap with respect to accessibility \cite{biagiMappingAccessabilityOpenStreetMap2020}.
This is important since OpenStreetMap depends on volunteers to gather and insert all the data. The easier
it is to add all the data required for accessibility, the more data will be made available, and the 
better an accessible route planner will be able to work.

Earlier research by \citeauthor{neisMeasuringReliabilityWheelchair2015a} 
\textcite{neisMeasuringReliabilityWheelchair2015a} studied the viability of using OpenStreetMap as source 
data. \textcite{dzaficImprovingWheelchairRoute2020} build
on this and proposes the use of a specialized route planning application in cooperation with a more advanced
wheelchair capable of e.g. climbing stairs. While an interesting thought, it is not something 
available to everyone. 10 years have passed since \citeauthor{neisMeasuringReliabilityWheelchair2015a}'s
original research and much might have changed in the meantime on the detail available in OpenStreetMap.

Literature shows a wide variety of approaches to accessible route planning. Each has trade-offs in 
scalability (can it easily be applied to other locations?), data quality (high quality scans or 
crowd sourced data). The use of OpenStreetMap offers a practical path to more widespread accessible 
route planning. While crowd sourcing could have a negative connotation, it also means that the source
data can be, and is, continuously improved.



\section*{Research methodology}

As we intend to build a route planner for less mobile people, we should first define our precise target
audience. There are various forms of reduced mobility: wheelchair bound people, people using crutches temporarily 
or permanently, being less mobile due to age, blindness, \dots. Do we take all forms into account, or do we focus on a certain subgroup?

Once we know our target audience, we need to define their needs for an \textit{accessible route}. Several 
aspects impact a route:

\begin{itemize}
  \item What kind of slope is acceptable?
  \item What is the impact of different kinds of surface: concrete, paving stones, grass, mud, cobblestones, etc.
  \item Is the walkway separated from the road?
  \item How wide is the walkway?
  \item Is there a curb, and how high is it?
  \item Are there stairs? Do they include a ramp?
\end{itemize}

Using these requirements, we can assign different weights to the different aspects. Given that different aspects
impact different ableness differently, weights could differ for different populations.

When building the route planner, it will need to take these weights into account based on a chosen profile.

\section*{Expected results}

Planning routes based on commonly and freely available mapping data and tailored to a user's specific needs
makes it easier for them to travel around independently.

Combine this with a presentation tailored to their needs (e.g. auditory guidance for blind people) and their
dependence on other people or guidance dogs should be reduced and make them more confident to travel alone.

% \section*{Timeline}

\printbibliography

% \appendix
% \section{Appendix A}

    
\end{document}