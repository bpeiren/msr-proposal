\documentclass{article}
\usepackage[british]{babel}
\usepackage{appendix}
\usepackage{hyperref}
\usepackage{csquotes}
% Use numeric citations and keep bibliography in citation order (appearance)
\usepackage[backend=biber,style=authoryear]{biblatex}
\addbibresource{msr-proposal.bib}

\author{Bart Peiren\thanks{Project repository: \url{https://github.com/bpeiren/msr-proposal}}}
\title{Research proposal:\\Explore and develop a route planner for people with limited mobility with OpenStreetMap}

\begin{document}

\maketitle


\section*{Introduction}

People with limited mobility encounter obstacles which are not taken into account by common route planners.
Common route planners create pedestrian routes for able bodied people. They do not take into account
obstacles which less mobile people might experience. This makes it harder for them to leave their home,
especially when travelling to unknown places.

While the data obtained via a LiDAR scan offers many possibilities, it also comes with two downsides: the whole 
environment where we want to plan a route needs to be mapped with special devices, and afterwards this data 
needs to be analysed so the useful data is extracted from it.

\citetitle{OpenStreetMap} already provides a map for most of the world. It might not be detailed 
enough for all use, but most likely those areas will not be LiDAR scanned either. 

If we build an accessible route planner based on OpenStreetMap, in a later phase the LiDAR data can be processed
and based on it, OpenStreetMap could be updated automatically. In this way the LiDAR data serves a broader goal
rather than the limited few applications.


\section*{Literature review}

\begin{displayquote}[\textcite{kaselourisAccessibilityRoutingModel2025}]
  Accessibility in urban environments is a fundamental right of people with disabilities and should be a 
  key component of an inclusive society.
\end{displayquote}

\textcite{kaselourisAccessibilityRoutingModel2025} describe the mapping of an Athenian municipality and 
the building of a mobile application for accessible route planning. The mapping was done by performing
street audits in a limited area and further processed to arrive at data ready to use in their algorithm.

\textcite{gharebaghiUserSpecificRoutePlanning2021} researched using a fuzzy approach to determine 
accessible routes for \textit{people with motor disabilities} in Quebec. This research focuses more on
the algorithm to obtain a route than the data the algorithm is based on.

In 2023-2024 Amsterdam Intelligence at the Municipality of Amsterdam Intelligence created a route planner for
people with reduced mobility. Their website \textcite{CreatingPersonalizedRoute} is no longer available, but 
fortunately their source code remains available on GitHub
\textcite{AmsterdamAITeamAccessible_Route_PlanningRepository} for further study. Their work was aimed at 
The Netherlands, so can not be widely reused abroad. The data source in their implementation are 3D point 
clouds, obtained via LiDAR scans (\textcite{fiawoPuntenwolkenVanGedetailleerde2022}). This method offers very 
detailed data, and makes precise
mapping of the environment possible, including walkway width, curb heights, as well as fixed obstacles 
such as bins and benches. While this is an ideal source, it is very resource intensive and might not be 
available for every location nor be kept up-to-date.

\textcite{cohenRoutePlanningBlind2021} describe an implementation of a route planner for blind people 
based on a weighted graph. Planning a route for less mobile people could be similar in idea, but with 
different weights.

\textcite{kocaaslanFindingOptimalRoute2024} performed a study in Turkey using the analytical hierarchy 
process to find accessible routes. They used existing maps where they assigned various criteria
a certain importance.

The data made available in OpenStreetMap is more limited than data available from LiDAR scans. It might also
be less detailed than other existing maps. However, it is freely and widely available, and as such more 
accessible. 
Though limited, it might be sufficient for our purpose. It offers a wide variety of functionality to map roads with 
their sidewalks. The degree of available detail in a given area decides how useful OSM could be for
accessible route planning in this area.

\textcite{biagiMappingAccessabilityOpenStreetMap2020} 
conducted a study on different techniques of adding data in OpenStreetMap with respect to accessibility.
This is important since OpenStreetMap depends on volunteers to gather and insert all the data. The easier
it is to add all the data required for accessibility, the more data will be made available, and the 
better an accessible route planner will be able to work.

Earlier research by 
\textcite{neisMeasuringReliabilityWheelchair2015a} studied the viability of using OpenStreetMap as source 
data. \textcite{dzaficImprovingWheelchairRoute2020} build
on this and proposes the use of a specialised route planning application in cooperation with a more advanced
wheelchair capable of e.g. climbing stairs. While an interesting thought, it is not something 
available to everyone. 10 years have passed since \citeauthor{neisMeasuringReliabilityWheelchair2015a}'s
original research and much might have changed in the meantime on the detail available in OpenStreetMap.

Literature shows a wide variety of approaches to accessible route planning. Each has trade-offs in 
scalability (can it easily be applied to other locations?), data quality (high quality scans or 
crowd sourced data). The use of OpenStreetMap offers a practical path to more widespread accessible 
route planning. While crowd sourcing could have a negative connotation, it also means that the source
data can be, and is, continuously improved.



\section*{Research methodology}

Developing an accessible route planner for people with limited mobility consists of multiple phases. To start,
we need to define our precise scope and target audience. There are various forms of reduced mobility: 
wheelchair bound people, people using crutches temporarily or permanently, the elderly, visually impaired people,
and many more. We need to decide whether we take multiple forms into account, or whether we focus on a certain subgroup.
To ensure that the planner is suited for real-world use, we need to have representatives of each subgroup to
cover their specific needs.

Essential to this research is the source data. Using OpenStreetMap, we ensure we have freely available mapping
for most locations. The level of detail varies from location to location, so this needs to be assessed for our
concrete case. Should the detail be insufficient, additional data might be gathered via multiple methods 
(manual charting, municipal data, etc.). This data will be fed back into OpenStreetMap to ensure that any
new data is not kept to this research, but is freely available as well. Checking the available data will 
consist of validations on the presence of sidewalk data, sidewalk width, slope data, surface type, presence
of stairs, and curb height.

With both our audience and map data available, we can define which criteria decide an \textit{accessible route}.
These criteria include:

\begin{itemize}
  \item What kind of slope is acceptable?
  \item What is the impact of different kinds of surface: concrete, paving stones, grass, mud, cobblestones, etc.
  \item Is the sidewalk separated from the road?
  \item How wide is the sidewalk?
  \item Is there a curb, and how high is it?
  \item Are there stairs? Do they include a ramp? Are they open to the side? Do they have handrails?
\end{itemize}

Not all of these criteria will be of equal importance to each subgroup, or even to each individual. To assign
appropriate weights to various profiles, we will need input from the representatives as well as user feedback.
This ensures that the route planner has a basic set of profiles which can be used, while individual users
can tailor it to their own needs and preferences.

Being able to personalize your profile is key since it helps to take into account the criteria which you
as an individual find important. The route planner will create routes taking the individual profile into 
account, which makes for a route best suited for one's specific needs.

The essence of the route planner will be the algorithm calculating the optimal route for a given profile, 
based on the weights assigned to this profile. A weighted graph will be implemented where each section of the 
route is assigned a certain weight based on the weights of the profile. This section weight shows how accessible
the section is for this profile. To decide the weights we could use multi criteria decision making.

The effectiveness of the route planner will be evaluated based on user testing with multiple participants from
each of our target groups. This testing will be done both by simulating routes as well as real world validation
of the suggested routes. Correctness of the routes and user satisfaction are key to having a successful route planner.

Given that we are working with sensitive data such as user disabilities and user location, privacy must be 
one of the foundations on which the route planner is built. This must be considered in every stage of data
collection and processing.


\section*{Expected results}

The goal of this research is to create a prototype iOS application able to plan accessible routes based on
the freely available data from OpenStreetMap tailored to the user's specific needs, making it easier for them
to travel around independently. Success is measured by user satisfaction. Is it easier to travel around 
independently than before? Are they more confident to travel around alone?

If successful this application can be made available publicly for people to try in their own neighbourhood.

Building this application will reveal the limitations of OpenStreetMap we run into. In future research one
could consider how to alleviate these limitations by improving OpenStreetMap which everyone benefits from.

Given that OpenStreetMap is crowd sourced, there is a fair chance that not all data which we require will
be available. Where we notice it is lacking, we will enrich it to suit our needs.

In other research one could investigate how LiDAR scans could provide additional data to OpenStreetMap.
This way everyone can benefit from the scanned data rather than it being restricted to one specific use case.


% \section*{Timeline}

\printbibliography

% \appendix
% \section{Appendix A}

    
\end{document}